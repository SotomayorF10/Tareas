\documentclass[12pt]{article}
\usepackage[utf8]{inputenc}
\usepackage{amsmath}
\usepackage{amssymb}
\usepackage{mathrsfs} %diferentes tipos de letras
\usepackage{amsmath}
\usepackage{amssymb}
\usepackage{latexsym}
\usepackage{mathrsfs}
\usepackage{xcolor}
\usepackage{dsfont}
\usepackage{amsfonts}
\usepackage{array}
\usepackage{fancyhdr}

%___________________________________________________-

%___________________________________________________-

 %encabezados y pie
 \pagestyle{fancy}
   \fancyhf{}
  %direcciones del encabezado
    \rhead{\Large{Tarea. Formulario}}
    \lhead{\Large{Sotomayor Fernando}}
     %pie
     \cfoot{}

     
   \title{{25 ecuaciones \\ para clase de computaci\'on}}
  
 \author{Sotomayor Fernando}
  \date{\today}
    


\begin{document}

\maketitle{\huge{F\'ormulas}}

\section{Fórmula Cuadrática: se usa para resolver ecuaciones cuadráticas. Es la solución de la ecuación cuadrática general.}
\subsection{$$x = \frac {-b \pm \sqrt {b^2 - 4ac}}{2a}$$}


 
\section{Fórmula Cuadrática para la Ecuación Cuadrática Simplificada: nos da una manera segura de resolver las ecuaciones cuadráticas de la forma 0 = ax 2 + bx + c. }

 \subsection{\textcolor{blue}$x_{1/2} = - \frac{p}{2}{ \pm \sqrt {\left( \frac{p}{2} \right)^2 - q}}$}
 
\section{Formato Vértice en Cúbicas: es el punto donde la función cambia de dirección.}

 \subsection{\textcolor{pink}$a x^3 + b x^2 + c x + d = 0$}
 
 \section{Fórmula de la Distancia dos puntos: herramienta útil para encontrar la distancia entre dos puntos que se pueden representar arbitrariamente}
 
 \subsection{\textcolor{red}$\sqrt{(x_1 - x_2)^2 + (y_1 - y_2)^2}$}
 
\section{Pendiente de una recta: se utiliza para calcular la inclinación de una recta con una curva.}

 \subsection{\textcolor{yellow}$m=\frac{y_2-y_1}{x_2-x_1}$}
 
\section{La ecuación de una recta en el plano cartesiano es de la
forma: donde a, b y c son números reales; a ≠ 0 ó b ≠ 0, y (x, y) representa un punto genérico de L”}

 \subsection{\textcolor{brown}$ax+by+c=0$}
 
 \section{El teorema de Fermat: es un teorema de análisis real.}
 
 \subsection{\textcolor{orange}$2^2^n+1$ −→ 2
2n + 1}
 
\section{Sucesión de Fibonacci: es en sí una sucesión matemática infinita.}

 \subsection{La sucesión $(x_n)$ definida por \textcolor{purple}
$$x_1=1,\quad x_2=1,\quad x_n=x_{n-1}+x_{n-2}\;\;(n>2)$$}

\section{Ecuaciones de Cauchy-Riemann: Análisis de funciones complejas de variable compleja}

 \subsection{\begin{equation*}
\left.
\begin{aligned}
u_x & = v_y\\
u_y & = -v_x
\end{aligned}
\right\}
\quad\text{Ecuaciones de Cauchy-Riemann}
\end{equation*}}


\section{Integral de una funci\'on: una operación donde, dada una función “f (x)”, permite determinar su función primitiva “F (x)”}

 \subsection{$\textcolor{green}{\int_0^\infty f(x)d(x) = g(x) + C}$}

\section{Ecuaciones de Maxwell:  nos permiten evaluar el comportamiento del campo electromagnético en una región del espacio o en un material, lo cual es la base del estudio de la electricidad y el magnetismo}

 \subsection{\begin{subequations} \label{eq:Maxwell} Ecuaciones de Maxwell: \begin{align} B'&=-\nabla \times E, \label{eq:MaxB} \\ E'&=\nabla \times B - 4\pi j, \label{eq:MaxE} \end{align} \end{subequations}}
 
\section{Leyes de Newton: analizar las fuerzas que actúan sobre un objeto y determinar así, su estado de movimiento. Esto tiene infinidad de aplicaciones prácticas}

\subsection{\begin{equation} \textcolor{violet}
F = ma \quad\text{Segunda ley de Newton}
\end{equation}}

\section{Volumen de una esfera: El volumen de una esfera es la cantidad de espacio que ocupa. }

 \subsection{\begin{equation}
\label{eq:esfera}
V=\frac{4}{3}\pi r^2
\end{equation}}

\section{Teo. de Fermat 1.1}

 \subsection{El teorema de Fermat establece que para $n > 2$, no hay enteros $x$,
$y$, $z$ que cumplan:
$$x^y + y^n = z^n$$}

\section{iFórmulas Trigonométricas Básicas:  sirven en triángulo rectángulos para relacionar sus lados con sus ángulos}

 \subsection{$\sin A = \frac {opp}{hyp} = \frac {a}{c} = (a/c)$}
 
 
 
\section{Interés Compuesto: El interés compuesto hace referencia al interés que va aportando a lo largo de los años la rentabilidad de tu inversión inicial.}

 \subsection{$Monto = Inicial \cdot \left( 1 + \frac {tasa}{períodos} \right) ^ {tiempo \cdot períodos}$}
 
\section{Formula del Vertice: Hallar el valor x del vértice }

 \subsection{\textcolor{violet}$f(x) = a(x - h)^2 + k$}
 
\section{Fórmula Cuadrática para la Ecuación Cuadrática Simplificada}

 \subsection{\textcolor{blue} $x_{1/2} = - \frac{p}{2}{ \pm \sqrt {\left( \frac{p}{2} \right)^2 - q}}$}
 
\section{Limites:  es la clave de toque que formaliza la noción intuitiva de aproximación hacia un punto concreto de una sucesión o una función}

 \subsection{$\lim_{x \to \infty} \left( \frac{1}{x} \right)$}
 
\section{Fuerza de gravedad}

 \subsection{\vec{F_{12}}=-G\dst\dfrac{m_1m_2}
{|r|^2}\vec{e_r}}
 
\section{Tercera ley de Newton:  nos explica las fuerzas de acción y reacción y nos enseña que son fuerzas que podemos encontrar en todos los cuerpos cuando éstos están en contacto unos con otros}

 \subsection{$$\vec {F}_{12}=-\vec {F}_{21}$$}
 
\section{Equilibrio estatico: está en equilibrio y no experimentará aceleración de fuerzas externas.}

 \subsection{$\dst\sum^n_{i=1} \vec {F_i}= m \cdot
\vec a=\vec 0

\dst\sum^n_{i=1} \vec {M_i}=
\dst\sum^n_{i=1}
\vec {r_i}\times \vec {F_i} =\vec 0$}
 
\section{Centro de masa: El centro de masas representa el punto en el que suponemos que se concentra toda la masa del sistema para su estudio}

 \subsection{$C_m=\dst \int_M r \dd m=\int_V
\rho r \dd V$}
 
\section{Teorema de steiner: Se encarga de establecer una evaluación precisa sobre el momento de inercia de un cuerpo extendido por el eje paralelo.}

 \subsection{$I_s=I_l+m\cdot r_{s-l}^2$}

\section{Periodo del oscilado armónico: Se denomina período al tiempo que tarda en producirse una oscilación completa, al tiempo que tarda en repetirse el movimiento.  }

 \subsection{$T=2\pi\sqrt{\dfrac{m}{k}}$}
 


\end{document}
